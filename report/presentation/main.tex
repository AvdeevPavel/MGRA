\documentclass[10pt,pdf,hyperref={unicode}]{beamer}
\usetheme{Antibes} % Beamer theme v 3.0
\usecolortheme{blue}
\usepackage[T2A]{fontenc}       %поддержка кириллицы
\usepackage[utf8]{inputenc}   %пока бибтех не дружит до конца с юникодом
\usepackage[english,russian]{babel}     %определение языков в документе
\usepackage{amssymb,amsmath}    %математика



%% Выбор шрифтов %%
\usefonttheme[onlylarge]{structurebold}

% Привычный шрифт для математических формул
\usefonttheme[onlymath]{serif}

% Более крупный шрифт для подзаголовков титульного листа
\setbeamerfont{institute}{size=\normalsize}


% Если используется последовательное появление пунктов списков на
% слайде (не злоупотребляйте в слайдах для защиты дипломной работы),
% чтобы еще непоявившиеся пункты были все-таки немножко видны.
\setbeamercovered{transparent}

%%% Сокращения %%%
% Синий цвет выделения
\setbeamercolor{color1}{bg=blue!60!black,fg=white}
\newcommand{\celcius}{\,^{\circ}\mathrm{C}}  %градус Цельсия
\newcommand{\grad}{\,^{\circ}} 


\title{MGRA(Multiple Genome Rearrangements and Ancestors) server}
\author{Авдеев Павел}
\institute{Algorithmic Biology Lab \\
    \vspace{0.7cm}
    Руководитель:  Максим Алексеев \\
    \vspace{0.7cm}
}
\date{
    Санкт-Петербург\\
    31 августа 2012г.
}

\begin{document}
\begin{frame}
  \maketitle
\end{frame}

\begin{frame}
	\begin{center}
		\only<1-4>{Здесь должен быть скучный текст.\\} 
		\only<2-4>{\huge{НО} \\}
		\only<3-4>{\normalsize{Лучше один раз увидеть, чем один раз услышать} \\}
		\only<4>{\large{поэтому...} \\}
	\end{center}
\end{frame}

\begin{frame}
	\frametitle{Благодарности}
	\begin{enumerate}
		\item Максиму Алексееву за терпение и помощь.
		\item Якову Сироткину
		\item Лаборатории за уникальную возможность хорошо провести лето.
	\end{enumerate}
\end{frame}

\begin{frame}
	\frametitle{Последний слайд}
	\begin{center}
		\Huge{\textbf{Спасибо за внимание!}}
	\end{center}
\end{frame}

\end{document}
