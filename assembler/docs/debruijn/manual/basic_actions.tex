\documentclass[a4paper,10pt]{article}


%opening
\title{EdgeGraph manual}
\author{Anton, Sergey, Nikoly}

\begin{document}

\maketitle

\section*{What is EdgeGraph?}
EdgeGraph is an implementation of DeBruijn graph with long sequences stored in its edges.
Implementation includes methods to change graph structure, iterate through graph and store coverage information.
EdgeGraph supports reverce-complement structure which means that each operation is performed simultaneously with 
certain element(vertex or edge) and reverse-complement element.
EdgeGraph was designed in such a way that it handles most of reverce-complement structure problems automatically.
Some external tools like tip clipping and bulge removal are implemented to be used with EdgeGraph interface.

In EdgeGraph each vertex corresponds to k-mer.
Each edge represents sequence of k+1-mers.
Thus length of edge is number of k+1-mers in it rather than number of nucleotides in sequence stored in the edge.
In order to forbid vertices to be reverse-complement to itself k should be chosen to be odd.

\section*{Classes Structure}
EdgeGraph consists of three basic structures: class EdgeGraph itself, class Vertex and class Edge.
Vertex and Edge classes have only private methods and one does not even need to know these classes exist.
Instead one should use methods of EdgeGraph with arguments of types EdgeGraph::EdgeId and EdgeGraph::VertexId which actually are
Edge* and Vertex*.

\section*{Inner Structure Of EdgeGraph}
EdgeGraph contains set of all VertexId.
Each vertex contains list of ids of outgoing edges and pointer to reverse-complement vertex.
Each edge contains Sequence of nucleotides it represents, pointer to reverse-complement edge, coverage and pointer to end vertex.
Thus memory usage of the whole structure is as fallows: for each vertex 2 pointers are stored(pointer to this vertex in set, pointer
to reverse-complement vertex and pointer in vector of outgoing edges) plus overhead for vector stored in vertex(44 bytes in total).
For each edge also 3 pointers(pointer from vertex, pointer to reverse-complement edge and pointer to end Vertex),
Sequence(17 bytes) and coverage are stored(45 in total). 

\section*{Basic Interface To Get Information From EdgeGraph}
\subsection*{Graph Methods}
\begin{itemize}
\item \texttt{EdgeGraph(size\_t k)} create empty graph with given value of k. Assertion of k being odd is performed.
\item \texttt{~EdgeGraph()} destructor also deletes all vertices and edges in graph. Graph destructor does not delete Handlers! 
The one responsible for Handler deletion is the one who added it.
\item \texttt{size\_t size()} returns number of vertices in the graph.
\item \texttt{EdgeGraph::VertexIterator begin()}

\texttt{EdgeGraph::VertexIterator end()}

These methods allow to iterate through all vertices of graph.
\item \texttt {size\_t k()}

\end{itemize}

\subsection*{Methods For Vertices}
\begin{itemize}
\item \texttt{vector<EdgeId> OutgoingEdges(VertexId v)} returns list of all outgoing edges. Vector of outgoing edges stored 
in v is returned as is
\item \texttt{vector<EdgeId> IncomingEdges(VertexId v)} returns list of all incoming edges. Be careful: each time this method 
is called resulting vector is created
\item \texttt{EdgeId OutgoingEdge(VertexId v, char nucl)} returns outgoing edge with given nucleotide at k-th position in sequence of
edge returned(which is the first position outgoing edges have different nucleotides in).
If such edge was not found method returns NULL.
\item \texttt{size\_t OutgoingEdgeCount(VertexId v)} returns the number of outgoing edges
\item \texttt{size\_t IncomingEdgeCount(VertexId v)} returns the number of incoming edges

\item \texttt{bool CheckUniqueOutgiongEdge(VertexId v)} returns true if vertex has only one outgoing edge and false otherwise
\item \texttt{EdgeId GetUniqueOutgoingEdge(VertexId v)} returns unique outgoing edge
If outgoing edge is not unique throws exception
\item \texttt{bool CheckUniqueIncomingEdge(VertexId v)} returns true if vertex has only one incoming edge and false otherwise
\item \texttt{EdgeId GetUniqueIncomingEdge(VertexId v)} returns unique incloming edge. 
If incoming edge is not unique throws exception
\item \texttt{VertexId Complement(VertexId v)} returns reverse-complement vertex id
\item \texttt{bool IsDeadEnd(VertexId v)} returns true if the edge has no outgoing edges and false otherwise
\item \texttt{bool IsDeadStart(VertexId v)} returns true if the edge has no incoming edges and false otherwise 
\item \texttt{Sequence VertexNucls(VertexId v)} returns Sequence of length k corresponding to this vertex. Resulting
Sequence is calculated form Sequences stored in edges. If vertex has no outgoing or incoming edges assert exception is
thrown.
\end{itemize}

\subsection*{Methods For Edges}
\begin{itemize}
\item \texttt{const Sequence\& EdgeNucls(EdgeId edge)} returns Sequence stored in the edge
\item \texttt{size\_t length(EdgeId edge)} returns length of the edge which is number of k+1-mers
\item \texttt{EdgeId Complement(EdgeId e)} returns complement edge
\item \texttt{VertexId EdgeStart(EdgeId edge)} returns start vertex of the edge
\item \texttt{VertexId EdgeEnd(EdgeId edge)} returns end vertex of the edge
\end{itemize}

\subsection*{Coverage Methods}
Coverage of edge in edgegraph is the number of times k+1-mers from the edge appeared in reads. Most methods deal with this 
definition of coverage. This coverage has a very useful property that it is additive and also this value is always integer. 
But the value that really represents quality of edge is average value of k+1-mer coverage over all k+1-mers in the edge.
Thus when one computes coverege the first value should be used and when one measures quality of edge average value should be used.
It is easy to distinguesh which coverage is meant in which of the methods: one of them is size\_t and the other is double.

\begin{itemize}
\item \texttt{void SetCoverage(EdgeId edge, size\_t cov)} sets coverage value of the edge.
\item \texttt{double coverage(EdgeId edge)} returns average coverage of the edge
\item \texttt{size\_t KPlusOneMerCoverage(EdgeId edge)} returns average coverage of the edge
\item \texttt{void IncCoverage(EdgeId edge, int toAdd)} increases coverage by given value
\item \texttt{void IncCoverage(EdgeId edge)} increases coverage by 1
\end{itemize}

\section*{Graph Modification Methods}
\begin{itemize}
\item \texttt{VertexId AddVertex()} adds new vertex to EdgeGraph and returns id of new Vertex.
Simultaneously another vertex is created and marked as reverse-complement to this one.
\item \texttt{void DeleteVertex(VertexId v)} deletes vertex and its complement. Asserts if this vertex has outgoing or incoming
edges.
\item \texttt{Edge* AddEdge(VertexId v1, VertexId v2, const Sequence \&nucls, size\_t coverage = 0)} creates edge with given parameters
and reverse-complement parameters. Sometimes these two edges coinside, In this case only one edge is created and marked as 
reverse-complement to itself
\item \texttt{void DeleteEdge(EdgeId edge)} removes edge and its reverse-complement
\item \texttt{EdgeId MergePath(const vector<EdgeId>\& path)} merges path into one edge. Asserts if such merge is not valid i.e.
one of inner vertices of path contains does not have unique outgoing or incoming edge
\item \texttt{pair<EdgeId, EdgeId> SplitEdge(EdgeId edge, size\_t position)} splits edge in two in given positionand creates new 
vertex in the middle of the edge
\item \texttt{void GlueEdges(EdgeId edge1, EdgeId edge2)} glues edge1 to edge2. Also glues its start and end vertices. Be careful:
such operation may cause inconsistency in Sequences in edges adgecent to vertex
\subsection*{Supporting Methods}
\item \texttt{void ForceDeleteVertex(VertexId v)} delete all outgoing and incoming edges and then delete vertex
\item \texttt{bool AreLinkable(VertexId v1, VertexId v2, const Sequence \&nucls)} returns true if nucls starts with k-mer from v1 and
ends with k-mer from v2.
\item \texttt{void Merge(EdgeId edge1, EdgeId edg2)} merges path of two edges
\item \texttt{void CompressVertex(VertexId v)} merges unque outgoing and unique incoming edge of the vertex.
\item \texttt{bool CanCompressVertex(VertexId v)} checks if vertex has unique incoming and unique ougoing edges.
\item \texttt{void CompressAllVertices()} finds all vertices with unique incloming and unique outgoing edges and compresses it.
\end{itemize}
\end{document}
